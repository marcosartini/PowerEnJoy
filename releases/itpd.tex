\documentclass{scrreprt}
\usepackage{listings}
\lstset{
  basicstyle=\ttfamily,
  columns=fullflexible,
  keepspaces=true,
}
\lstset{language=Java,
  showspaces=false,
  showtabs=false,
  breaklines=true,
  showstringspaces=false,
  breakatwhitespace=true,
  commentstyle=\color{green},
  keywordstyle=\color{blue},
  stringstyle=\color{red},
  basicstyle=\ttfamily,
  moredelim=[il][\textcolor{grey}]{$$},
  moredelim=[is][\textcolor{grey}]{\%\%}{\%\%}
}
\usepackage{underscore}
\usepackage{booktabs}
\usepackage{tabularx}
\usepackage{array}
%\usepackage[bookmarks=true]{hyperref}
\usepackage{varioref}
\usepackage{hyperref}
\usepackage[utf8]{inputenc}
\usepackage[english]{babel}
\usepackage{verbatim}
\usepackage{enumitem}
\usepackage{subfig}
\usepackage{graphicx}
\usepackage{url}
\usepackage{multirow}


\hypersetup{
    bookmarks=false,    % show bookmarks bar?
    pdftitle={Design Document},    % title
    pdfauthor={Marco Sartini, Daniele Riva},                     % author
    pdfsubject={Power EnJoy},                        % subject of the document
    pdfkeywords={software engineering, car, power, project}, % list of keywords
    colorlinks=true,       % false: boxed links; true: colored links
    linkcolor=blue,       % color of internal links
    citecolor=black,       % color of links to bibliography
    filecolor=black,        % color of file links
   % urlcolor=purple,        % color of external links
    linktoc=page            % only page is linked
}%
\def\myversion{1.0 }
\def\version{1.0}
%\date{28/10/2016}
\titlehead{\centering\includegraphics[width=6cm]{Logo_Politecnico_Milano.png}}
\title{Integration Test Plan Document\\for\\PowerEnJoy}
\date{\today\\\bigskip version \version}
\author{Daniele Riva\thanks{matr. 875154}\and Marco Sartini\thanks{matr. 877979}}

\begin{document}
\pagestyle{headings}
\maketitle

\tableofcontents

\chapter{Introduction}

\section{Purpose}
This Integration Test Plan Document has the purpose to establish an acceptable program of tests to be performed on the developed modules and on the overall system.

The aim of the tests is to solicit the system and try to find incompleteness, bugs, modules compatibility problems, also minimizing the possibility to obtain critical and wrong states.
Specifically, this document gathers a number of integration tests to attempt to ensure the rightness of the behaviors among components.

\section{Scope}
The project \emph{Power EnJoy} is a platform based on mobile and web application thought to offer a car sharing service with electrical powered cars. 
\section{Definitions, acronyms, abbreviations}
\begin{description}
\item[RASD] requirements analysis and specification document;
\item[DD] design document;
\item[DBMS] data base management system;
\item[Arquillian] is a Java Integration Test framework. Details at \url{http://www.arquillian.org};
\item[Mockito] is a mocking framework. Details at \url{http://site.mockito.org};
\item[JUnit] is a simple framework to write repeatable tests. Based on xUnit architecture. Details at \url{http://junit.org}.
\end{description}
\section{Reference documents}
\begin{itemize}
\item RASD v1.0 available at \url{https://github.com/marcosartini/PowerEnJoy/blob/master/releases/rasdPowerEnJoy.pdf}
\item DD v1.0 available at \url{https://github.com/marcosartini/PowerEnJoy/blob/master/releases/dd.pdf}
\end{itemize}

\chapter*{Revision history}
\begin{center}
    \begin{tabular}{lccc}
        \toprule
	   \textbf{ Name }& \textbf{ Date  }& \textbf{ Reason For Changes }& \textbf{ Version}\\
        \midrule
	     Marco e Daniele & 10/01/2017 & Starting & 1.0\\
	\bottomrule
    \end{tabular}
\end{center}

\chapter{Integration strategy}

\section{Entry criteria}
Before the integration testing phase begin, all the modules have to be properly unity tested. This means that every single module should work right independently by the others, that is, it should return the expected results related to the input provided.
To speed up the starting of this step, it is not mandatory to have all the modules fully developed, but it is sufficient to dispose of the subparts that compose a thread. In some cases, although, a ordered schedule is requested.
In particular, all the methods listed and described in the DD should be tested.

\section{Elements to be integrated}
The components to be integrated are:

\begin{enumerate}
\item MapManager, to be integrated with the CarsManager;
\item RentalManager, to be integrated with CarsManager, KeepAsideManager, OnBoardSystemManager, UserManager, PaymentManager and DataManager;
\item OnBoardSystemManager, to be integrated with the RentalManager;
\item ReservationManager, to be integrated with CarsManager, PaymentManager, UserManager RentalManager and DataManager;
\item KeepAsideManager, to be integrated with RentalManager, CarsManager and DataManager;
\item IssueManager, to be integrated with the UserManager, CarsManager and DataManager.

\end{enumerate}

The other components are standalone, or at least they only serve the other components.
To be precise, the ones which do not consult nor delegate to other components except the Data Manager are: UserManager, SettingsManager and PaymentManager.

\section{Integration testing strategy}
The modules are widely related each other. This means that the more traditional testing approaches are a bit overcharged by extra stubs and drivers.
This consideration leads us to invest in a more efficient test process.

Then, we opt for the \emph{thread approach}.

We adopt also a critical-module-first approach before actually start the thread testing, to proper check the rightness of the interaction between DBMS and DataManager, between all the components and the DataManager.

Benefits granted by the thread approach allows, as soon as the methods and classes related to a thread are ready developed, to be immediately tested. It is not necessary that a component is fully developed, but it is enough that are developed the involved subparts in the thread.

In addition, less if none stubs and drivers are required and if a component in a thread doesn't work, it is rational to exclude the subparts already tested in the other threads simplifying the resolution.

\section{Sequence of function integration}
\subsection{DBMS and DataManager}
This test should be the first to be performed because the entire system is basically based on the data manipulation. All the thread considered below write and read data from the database through the DataManager.
\begin{center}
		\includegraphics[width=\textwidth]{./thread/DataComponentDBMS-Testing.pdf}
\end{center}

To test the DataManager, the component will be represented by drivers, one for each of them.
The drivers will perform typical queries to check the correctness of the results.
\begin{center}
		\includegraphics[width=\textwidth]{./thread/DataComponent-Testing.pdf}
\end{center}

\subsection{Threads}
Every identified thread is briefly described to clarify the its purpose and which components methods are involved.
In the image, the colors stress the parts of the components which contribute to build up a thread (user-visible functionality).
Some methods are shared by more than one thread, because many thread might need to employ the same methods.

Colors caption:
\begin{description}
\item[yellow] indicates the Rental functionality;
\item[orange] indicates the Reservation functionality;
\item[dark blue] indicates the End of the reservation functionality;
\item[pink] indicates the timeout triggered end of reservation functionality;
\item[green] indicates the Keep Aside functionality;
\item[red] indicates the start Rental functionality;
\item[light blue] indicates the end of the rental and related billing and payment functionality;
\item[brown] indicates the Map functionality;
\item[gray and rose] indicate the Issue functionality;
\end{description}

\begin{center}
		\includegraphics[width=\textwidth]{./thread/fullView.pdf}
\end{center}

Threads do no need to be tested and verified in a particular order because of the inherent feature of independence, but we suggest to adopt the following:
\begin{enumerate}
\item Issues;
\item Map;
\item Rental;
\item Reservation;
\item Start rental;
\item End reservation;
\item Keep Aside;
\item End rental.
\end{enumerate}

Note on the \emph{OnBoardSystemManager}: this component, despite in the reality will be installed on the cars, is tested as it were local, by setting up the test environment at this purpose. This because test the real car means to have the car ready, and also the communication infrastructure working.

%It is however allowed to directly mock the OnBoardSystemManager to proceed with the thread tests, 

\subsubsection{Issues}
The two components to be integrated are the \emph{IssueManager} and the \emph{CarsManager}.

A request to add an issue is sent by the user: the \emph{addIssue} method is activated, which is in charge to store an issue in the database, then it involves the \emph{CarsManager} who is in charge to update the state of the car (which is received as parameter in the original request).

If no exceptions are raised, the \emph{IssueManager} returns successfully to the caller.

\begin{center}
		\includegraphics[width=\textwidth]{./thread/00notify.pdf}
\end{center}

\subsubsection{Map}
It integrates with the \emph{CarsManager} while building the map, to get the only available cars to put into that map.

\begin{center}
		\includegraphics[width=\textwidth]{./thread/01map.pdf}
\end{center}

\subsubsection{Rental}
When a car has been chosen for a rental, and is confirmed, a request to apply for a rental is sent by the client. It is activated the \emph{RentalManager} via the \emph{addRental} method: the method is in charge to verify the availability of the car, thanks to the \emph{CarsManager} respective query method; then in case of positive answer, the \emph{CarsManager} is asked to update the car state; at the end, if no raised exception, the component generates an unlock code which is annotated in the \emph{OnBoardSystemManager} via the \emph{setUnlockCode}. Again if no raised exceptions, the unlock code is also returned to the client.

\begin{center}
		\includegraphics[width=\textwidth]{./thread/02rental.pdf}
\end{center}

\subsubsection{Unlock \& Start}
This thread is launched by the request of verifying the code. Once the code locally matches, the OnBoardSystem opens the door and listens for the engine to power on. In that moment, a request to start a rental is sent to the server, who refers to the \emph{RentalManager}. The \emph{RentalManager} stores on the database the data and it is ready to report the charging fees to the user.

\begin{center}
		\includegraphics[width=\textwidth]{./thread/03start.pdf}
\end{center}

\subsubsection{Keep aside}
When a ``keep aside'' request is delivered to the server, the \emph{KeepAsideManager} is activated, via the \emph{addKeepAside} method. That method contacts the \emph{CarsManager} to update the state of the car and, if no exceptions, starts the ``keep aside'' interval storing data in the database. If no exceptions, it returns to the \emph{OnBoardSystemManager}, which cares to check the car to be empty and locks the doors.
When a keep aside is going to be stopped, the user send a request to stop the ``keep aside'': the \emph{KeepAsideManager} is involved via the method \emph{stopKeepAside}which refers to the \emph{CarsManager} to get the state updated and if no exceptions, the doors are unlocked by the \emph{OnBoardSystemManager}.

\begin{center}
		\includegraphics[width=\textwidth]{./thread/04keep.pdf}
\end{center}

\subsubsection{End rental}
When a rental ends, the \emph{OnBoardSystemManager} it is unawakened because the engine turn off, then it checks the passengers and is mandatory that the number is 0. So it locks the doors and invokes the \emph{RentalManager} to end the rental via the \emph{endRental} method. The \emph{RentalManager} checks and eventually assigns discounts invoking its method \emph{assignDiscount}, then the \emph{PaymentManager} is involved to provide a bill and interact with the external Payment Handler.

Furthermore, the car state is updated by the \emph{CarsManager}. There is a full return from the stack to reassure the \emph{OnBoardSystemManager}.

\begin{center}
		\includegraphics[width=\textwidth]{./thread/05end.pdf}
\end{center}

\subsubsection{Reservation}
A car is chosen by the user, and is given to the \emph{ReservationManager} who cares to add the reservation and to start the timeout. The \emph{CarsManager} is at this point involved to have the car state updated; also the \emph{UserManager} is involved to have the user status updated. At the end, the \emph{ReservationManager} triggers the timeout.

\begin{center}
		\includegraphics[width=\textwidth]{./thread/07res.pdf}
\end{center}

\subsubsection{End reservation \& start rental}
Case 1: A user decides to regularly begin a rental of his reserved car.

Case 2: If the reservation timeout expires, the reservation ends, so the \emph{ReservationManager} will free the car from the reservation and will proceed to bill. 

\begin{center}
		\includegraphics[width=\textwidth]{./thread/06endRes.pdf}
\end{center}
\begin{center}
		\includegraphics[width=\textwidth]{./thread/08timeout.pdf}
\end{center}

\chapter{Individual steps and test description}
A schematic description of the test cases to be performed in order to meet the objectives of the integration test phase.

\section{Data Manager}

\begin{center}
\begin{tabularx}{\columnwidth}{>{\bfseries}lX}
\toprule
Test case identifier & ID0T1\\
\midrule
Test items & DataManager $\longrightarrow$ DBMS\\
\midrule
Input specification & Queries ( insert, update, delete)  on database tables \\
\midrule
Output specification & Correct result of each query\\
\midrule
Environmental needs & DBMS\\
\bottomrule
\end{tabularx}
\end{center}


\begin{center}
\begin{tabularx}{\columnwidth}{>{\bfseries}lX}
\toprule
Test case identifier & I0T1\\
\midrule
Test items & Driver UserManager $\longrightarrow$ DataManager\\
\midrule
Input specification & Calls method of UserManager to add new user or update data user \\
\midrule
Output specification &  DataManager generates queries on User table \\
\midrule
Environmental needs & UserManager driver\\
\bottomrule
\end{tabularx}
\end{center}


\begin{center}
\begin{tabularx}{\columnwidth}{>{\bfseries}lX}
\toprule
Test case identifier &I0T2 \\
\midrule
Test items & Driver CarsManager $\longrightarrow$ DataManager\\
\midrule
Input specification & Calls method of CarsManager to update cars State \\
\midrule
Output specification &  DataManager generates queries on Cars table\\
\midrule
Environmental needs & CarsManager driver\\
\bottomrule
\end{tabularx}
\end{center}


\begin{center}
\begin{tabularx}{\columnwidth}{>{\bfseries}lX}
\toprule
Test case identifier & I0T3\\
\midrule
Test items & Driver RentalManager $\longrightarrow$ DataManager\\
\midrule
Input specification & Calls method of RentalManager to add new rental or update data of existing rental \\
\midrule
Output specification &  DataManager generates queries on Rental table \\
\midrule
Environmental needs & RentalManager driver\\
\bottomrule
\end{tabularx}
\end{center}

\begin{center}
\begin{tabularx}{\columnwidth}{>{\bfseries}lX}
\toprule
Test case identifier & I0T4\\
\midrule
Test items & Driver ReservationManager $\longrightarrow$ DataManager\\
\midrule
Input specification & Calls method of ReservationManager to add new reservation or update data of existing reservation \\
\midrule
Output specification & DataManager generates queries on Rental table \\
\midrule
Environmental needs & ReservationManager driver\\
\bottomrule
\end{tabularx}
\end{center}



\begin{center}
\begin{tabularx}{\columnwidth}{>{\bfseries}lX}
\toprule
Test case identifier & I0T5\\
\midrule
Test items & Driver KeepAsideManager $\longrightarrow$ DataManager\\
\midrule
Input specification & Calls method of KeepAsideManager to add new keepAside of a rental or update data of existing keepAside \\
\midrule
Output specification & DataManager generates queries onKeepAside table \\
\midrule
Environmental needs & KeepAsideManager driver\\
\bottomrule
\end{tabularx}
\end{center}

\begin{center}
\begin{tabularx}{\columnwidth}{>{\bfseries}lX}
\toprule
Test case identifier & I0T6\\
\midrule
Test items & Driver PaymentManager $\longrightarrow$ DataManager\\
\midrule
Input specification & Calls method of PaymentManager to add a new bill \\
\midrule
Output specification & DataManager generates queries on Payment table \\
\midrule
Environmental needs & PaymentManager driver\\
\bottomrule
\end{tabularx}
\end{center}


\begin{center}
\begin{tabularx}{\columnwidth}{>{\bfseries}lX}
\toprule
Test case identifier & I0T7\\
\midrule
Test items & Driver IssueManager $\longrightarrow$ DataManager\\
\midrule
Input specification & Calls method of IssueManager to add a new issue or update an issue state \\
\midrule
Output specification & DataManager generates queries on Payment table \\
\midrule
Environmental needs & IssueManager driver\\
\bottomrule
\end{tabularx}
\end{center}

\begin{center}
\begin{tabularx}{\columnwidth}{>{\bfseries}lX}
\toprule
Test case identifier & I0T8\\
\midrule
Test items & Driver SettingsManager $\longrightarrow$ DataManager\\
\midrule
Input specification & Calls methods of SettingsManager to manage prices \\
\midrule
Output specification & DataManager generates queries on Settings table \\
\midrule
Environmental needs & SettingsManager driver\\
\bottomrule
\end{tabularx}
\end{center}

\section{Issue}

\begin{center}
\begin{tabularx}{\columnwidth}{>{\bfseries}lX}
\toprule
Test case identifier & I1T1\\
\midrule
Test items & Issue Manager $\longrightarrow$ Car Manager\\
\midrule
Input specification & Call the addIssue method\\
\midrule
Output specification & The issue is added in the queue of the issue to solve; the car state is changed in ... \\
\bottomrule
\end{tabularx}
\end{center}

Below the description of the involved method calls.

\begin{center}
\begin{tabularx}{\columnwidth}{>{\bfseries}XX}
\toprule
\multicolumn{2}{>{\bfseries}c}{\textit{IssueManager $\longrightarrow$  addIssue(user, car, description, phoneNumber)}}\\
\toprule

One or more parameters are null & A NullArgumentException is raised\\
\midrule
Description is a empty string & An InvalidArgumentException is raised, the problem must be specified\\
\midrule
Phone number is an incorrect number & An InvalidArgumentException is raised \\
\midrule
Set of valid parameter & Creates an instance of issue with IssueState as \emph{ToSolve} and then writes it in the database \\

\bottomrule
\end{tabularx}
\end{center}

\begin{center}
\begin{tabularx}{\columnwidth}{>{\bfseries}XX}
\toprule
\multicolumn{2}{>{\bfseries}c}{\textit{CarsManager $\longrightarrow$  setState(car, state)}}\\
\toprule

One or more parameters are null & A NullArgumentException is raised \\
\midrule
Incorrect parameter: a car or a state that not exists & An InvalidArgumentException is raised \\
\midrule
Valid Parameters & Car state is changed into the new state and the car record in the database is updated \\
\bottomrule
\end{tabularx}
\end{center}

\subsection{Maintenance solve issues}

\begin{center}
\begin{tabularx}{\columnwidth}{>{\bfseries}lX}
\toprule
Test case identifier & I1T2\\
\midrule
Test items & Issue Manager $\longrightarrow$ Car Manager\\
\midrule
Input specification & A call to setIssueState\\
\midrule
Output specification & The car state is updated accordingly to the issue outcome\\
\bottomrule
\end{tabularx}
\end{center}

\begin{center}
\begin{tabularx}{\columnwidth}{>{\bfseries}XX}
\toprule
\multicolumn{2}{>{\bfseries}c}{\textit{IssueManager $\longrightarrow$ setIssueState(issue, man, state)}}\\
\toprule

One or more parameter are null & A NullArgumentException is raised\\
\midrule
Invalid parameter: the issue is already solved or is being solved & An IllegalIssueException is raised\\
\midrule
Invalid parameter: the man is already busy in another work & An IllegalIssueException is raised\\
\midrule
Valid parameter & Assigns the state parameter (InResolution or Solved) to the issue state and updates the database\\
\bottomrule
\end{tabularx}
\end{center}

\section{Map}

\begin{center}
\begin{tabularx}{\columnwidth}{>{\bfseries}lX}
\toprule
Test case identifier & I2T1\\
\midrule
Test items & MapManager $\longrightarrow$ CarsManager\\
\midrule
Input specification & Call to the generateMap method\\
\midrule
Output specification & The generated map contains the available cars, if any \\
\bottomrule
\end{tabularx}
\end{center}

Below the description of the involved method calls.

\begin{center}
\begin{tabularx}{\columnwidth}{>{\bfseries}XX}
\toprule
\multicolumn{2}{>{\bfseries}c}{\textit{MapManager $\longrightarrow$ searchAvailableCar(listCar)}}\\
\toprule

Parameter is null & A NullArgumentException is raised\\
\midrule
Empty List & Returns empty list, no cars are available\\
\midrule
Valid parameter: list with no available cars & Returns a empty list\\
\midrule
Valid parameter: list contains available cars & Returns a list with only available cars\\
\bottomrule
\end{tabularx}
\end{center}

\begin{center}
\begin{tabularx}{\columnwidth}{>{\bfseries}XX}
\toprule
\multicolumn{2}{>{\bfseries}c}{\textit{MapManager $\longrightarrow$ generateMap(listCar,position)}}\\
\toprule

One or more parameter are null & A NullArgumentException is raised\\
\midrule
Empty List & An IllegalMapException is raised\\
\midrule
A invalid position & An IllegalArgumentException is raised\\
\midrule
Valid parameters: no empty List and valid position & Returns a list of the available cars into the default distance\\
\bottomrule
\end{tabularx}
\end{center}

\section{Rental}

\begin{center}
\begin{tabularx}{\columnwidth}{>{\bfseries}lX}
\toprule
Test case identifier & I3T1\\
\midrule
Test items & RentalManager $\longrightarrow$ CarsManager, UserManager, OnBoardSystemManager\\
\midrule
Input specification & Call to the addRental method\\
\midrule
Output specification & The availability of the car is checked and the setState is called, then the user is set to busy, an unlock code is generated and sent to the car.\\
\bottomrule
\end{tabularx}
\end{center}

Below the description of the involved method calls.

\begin{center}
\begin{tabularx}{\columnwidth}{>{\bfseries}XX}
\toprule
\multicolumn{2}{>{\bfseries}c}{\textit{RentalManager $\longrightarrow$  addRental(user, car)}}\\
\toprule

One or more parameters are null & A NullArgumentException is raised \\
\midrule
Car is not available (checkAvailability returns false ) & An InvalidRequestException is raised, the car is unavailable\\
\midrule
User is already busy & An InvalidRequestException is raised, user is renting or reserving a car in that moment \\
\midrule
Set of valid parameter & Creates an instance of new rental; the car state is changed (from available state to rental state), it generates unlock code and then the rental is written to the database \\
\bottomrule
\end{tabularx}
\end{center}

\begin{center}
\begin{tabularx}{\columnwidth}{>{\bfseries}XX}
\toprule
\multicolumn{2}{>{\bfseries}c}{\textit{RentalManager $\longrightarrow$  generateUnlockCode(rental)}}\\
\toprule

Valid Parameter & Generates a code to unlock the car and associates it to the rental \\
\midrule
Parameter is null & A NullArgumentException is raised\\
\midrule
Incorrect parameter: finished rental & An InvalidArgumentException is raised\\
\bottomrule
\end{tabularx}
\end{center}

\begin{center}
\begin{tabularx}{\columnwidth}{>{\bfseries}XX}
\toprule
\multicolumn{2}{>{\bfseries}c}{\textit{CarsManager $\longrightarrow$  checkAvailability(car)}}\\
\toprule

Parameter is null & A NullArgumentException is raised \\
\midrule
Car state is not `AVAILABLE' & Function returns False \\
\midrule
Car state is `AVAILABLE' & Function returns True \\
\bottomrule
\end{tabularx}
\end{center}

\begin{center}
\begin{tabularx}{\columnwidth}{>{\bfseries}XX}
\toprule
\multicolumn{2}{>{\bfseries}c}{\textit{UserManager $\longrightarrow$  setBusy(user, state)}}\\
\toprule

One or more parameters are null & A NullArgumentException is raised \\
\midrule
Valid parameter & If state=true, it sets the user busy flag to \emph{true} else if state = false, it sets the user busy flag to \emph{false} \\
\bottomrule
\end{tabularx}
\end{center}

\begin{center}
\begin{tabularx}{\columnwidth}{>{\bfseries}XX}
\toprule
\multicolumn{2}{>{\bfseries}c}{\textit{OnBoardSystemManager $\longrightarrow$  setUnlockCode(code)}}\\
\toprule

Parameter is null & A NullArgumentException is raised \\
\midrule
Incorrect parameter: empty String code & An InvalidArgumentException is raised\\
\midrule
Valid parameter & Sets code to unlock the car\\
\bottomrule
\end{tabularx}
\end{center}

\section{Unlock \& Start rental}

\begin{center}
\begin{tabularx}{\columnwidth}{>{\bfseries}lX}
\toprule
Test case identifier & I4T1\\
\midrule
Test items & OnBoardSystemManager $\longrightarrow$ RentalManager;  RentalManager $\longrightarrow$ PaymentManager;\\
\midrule
Input specification & A code is injected in the OnBoardSystemManager, to have the code checked. Power on the engine.\\
\midrule
Output specification & If the code is correct, the doors unlock and when the engine powers on the rental starts.\\
\bottomrule
\end{tabularx}
\end{center}

Below the description of the involved method calls.

\begin{center}
\begin{tabularx}{\columnwidth}{>{\bfseries}XX}
\toprule
\multicolumn{2}{>{\bfseries}c}{\textit{OnBoardSystemManager $\longrightarrow$  checkCode(code)}}\\
\toprule

Parameter is null & A NullArgumentException is raised\\
\midrule
Code is incorrect (not equal to already saved code) & Returns false: unlock code is wrong\\
\midrule
Valid parameter & Returns true: code saved and code parameter are equal\\
\bottomrule
\end{tabularx}
\end{center}

\begin{center}
\begin{tabularx}{\columnwidth}{>{\bfseries}XX}
\toprule
\multicolumn{2}{>{\bfseries}c}{\textit{OnBoardSystemManager $\longrightarrow$  setLockDoors(state)}}\\
\toprule

Parameter is null & A NullArgumentException is raised\\
\midrule
State value: False & Sets the doors locked (false); car can't be opened\\
\midrule
State value: True & Sets the doors unlocked (true); car is opened\\
\bottomrule
\end{tabularx}
\end{center}

\begin{center}
\begin{tabularx}{\columnwidth}{>{\bfseries}XX}
\toprule
\multicolumn{2}{>{\bfseries}c}{\textit{OnBoardSystemManager $\longrightarrow$  checkEngine()}}\\
\toprule

No parameter & Simulates the engine state: if the engine state is set to ``run'' it returns \emph{true} else it returns \emph{false}\\

\bottomrule
\end{tabularx}
\end{center}

\begin{center}
\begin{tabularx}{\columnwidth}{>{\bfseries}XX}
\toprule
\multicolumn{2}{>{\bfseries}c}{\textit{RentalManager $\longrightarrow$  getRental(car)}}\\
\toprule

Parameter is null & A NullArgumentException is raised\\
\midrule
Invalid parameter: the car is not involved in any rental & An IllegalArgumentException is raised\\
\midrule
Valid parameter & Gets the rental ongoing with that car  \\
\bottomrule
\end{tabularx}
\end{center}

\begin{center}
\begin{tabularx}{\columnwidth}{>{\bfseries}XX}
\toprule
\multicolumn{2}{>{\bfseries}c}{\textit{RentalManager $\longrightarrow$  startRental(rental)}}\\
\toprule

Parameter is null & A NullArgumentException is raised\\
\midrule
Invalid parameter: ended rental & An IllegalArgumentException is raised\\
\midrule
Valid parameter & Sets the time when the rental starts and the value stored in the database is updated\\
\bottomrule
\end{tabularx}
\end{center}

\begin{center}
\begin{tabularx}{\columnwidth}{>{\bfseries}XX}
\toprule
\multicolumn{2}{>{\bfseries}c}{\textit{PaymentManager $\longrightarrow$  getCurrentCharging (rental)}}\\
\toprule

Parameter is null & A NullArgumentException is raised\\
\midrule
Invalid parameter: rental is already finished & An IllegalArgumentException is raised \\
\midrule
Valid parameter & Gets total charging of ongoing rental  (included all keepAsides and discounts) \\
\bottomrule
\end{tabularx}
\end{center}

\section{Keep aside}

\begin{center}
\begin{tabularx}{\columnwidth}{>{\bfseries}lX}
\toprule
Test case identifier & I5T1\\
\midrule
Test items & KeepAsideManager $\longrightarrow$ RentalManager, CarsManager;  RentalManager $\longrightarrow$ OnBoardSystemManager;\\
\midrule
Input specification & Call to the addKeepAside method\\
\midrule
Output specification & A keep aside is applied to the car. and the doors are locked\\
\bottomrule
\end{tabularx}
\end{center}

Below the description of the involved method calls.

\begin{center}
\begin{tabularx}{\columnwidth}{>{\bfseries}lX}
\toprule
Test case identifier & I5T2\\
\midrule
Test items & KeepAsideManager $\longrightarrow$ RentalManager, CarsManager;  RentalManager $\longrightarrow$ OnBoardSystemManager;\\
\midrule
Input specification & Call to the stopKeepAside method\\
\midrule
Output specification & The doors are unlocked, and the  car is ready to restart a rental.\\

\bottomrule
\end{tabularx}
\end{center}

\begin{center}
\begin{tabularx}{\columnwidth}{>{\bfseries}XX}
\toprule
\multicolumn{2}{>{\bfseries}c}{\textit{KeepAsideManager $\longrightarrow$  addKeepAside(rental)}}\\
\toprule

Parameter is null & A NullArgumentException is raised\\
\midrule
Invalid parameter: keep aside is not allowed for this rental at this moment & An IllegalKeepAsideException is raised \\
\midrule
Invalid parameter: checkKeepAsideOnGoing returns false & An IllegalKeepAsideException is raised, a keep aside is ongoing\\
\midrule
Valid parameter & Creates a new instance of keepAside and adds it to the rental, then the keepAside is written to the database\\
\bottomrule
\end{tabularx}
\end{center}

\begin{center}
\begin{tabularx}{\columnwidth}{>{\bfseries}XX}
\toprule
\multicolumn{2}{>{\bfseries}c}{\textit{KeepAsideManager $\longrightarrow$  startKeepAside(rental)}}\\
\toprule

Parameter is null & A NullArgumentException is raised\\
\midrule
Invalid Parameter : rental has no keepAside to start, rental has only finished keepaside & A IllegalKeepAsideException is raised\\
\midrule
Valid Parameter & Sets time when keepAside started and the value in the database is updated\\

\bottomrule
\end{tabularx}
\end{center}

\begin{center}
\begin{tabularx}{\columnwidth}{>{\bfseries}XX}
\toprule
\multicolumn{2}{>{\bfseries}c}{\textit{KeepAsideManager $\longrightarrow$  stopKeepAside(rental)}}\\
\toprule

Parameter is null & A NullArgumentException is raised\\
\midrule
Invalid Parameter : rental has no keepAside ongoing, rental has only finished keepaside & A IllegalKeepAsideException is raised\\
\midrule
Valid parameter & Set time that keepAside ends, set keepaside as finished and database value is updated\\

\bottomrule
\end{tabularx}
\end{center}

\begin{center}
\begin{tabularx}{\columnwidth}{>{\bfseries}XX}
\toprule
\multicolumn{2}{>{\bfseries}c}{\textit{RentalManager $\longrightarrow$ checkKeptAsideOnGoing(rental)}}\\
\toprule

Parameter is null & A NullArgumentException is raised\\
\midrule
Valid parameter	 & Return true if there is a keepaside on going, otherwise return false\\

\bottomrule
\end{tabularx}
\end{center}

\begin{center}
\begin{tabularx}{\columnwidth}{>{\bfseries}XX}
\toprule
\multicolumn{2}{>{\bfseries}c}{\textit{OnBoardSystemManager $\longrightarrow$  checkNobodyInTheCar()}}\\
\toprule

No parameter & If there is nobody in the car, calls the \emph{lockDoors}, otherwise it does nothing\\

\bottomrule
\end{tabularx}
\end{center}

\section{End rental}

\begin{center}
\begin{tabularx}{\columnwidth}{>{\bfseries}lX}
\toprule
Test case identifier & I6T1\\
\midrule
Test items & OnBoardSystemManager $\longrightarrow$ RentalManager; RentalManager $\longrightarrow$ PaymentManager, CarsManager;\\
\midrule
Input specification & The engine of the car is powered off\\
\midrule
Output specification & The doors are locked, the bill is stored\\
\midrule
Environmental needs & The stub to simulate the Payment Handler\\
\bottomrule
\end{tabularx}
\end{center}

Below the description of the involved method calls.

\begin{center}
\begin{tabularx}{\columnwidth}{>{\bfseries}XX}
\toprule
\multicolumn{2}{>{\bfseries}c}{\textit{RentalManager $\longrightarrow$  endRental(rental)}}\\
\toprule

Parameter is null & A NullArgumentException is raised\\
\midrule
Valid parameter & Calls the \emph{assignDiscount} function and assigns the value to the rental, then sets the time when rental is ended and updates the database value.

Then calls the method to update the car state and to compute the final price\\
\bottomrule
\end{tabularx}
\end{center}

\begin{center}
\begin{tabularx}{\columnwidth}{>{\bfseries}XX}
\toprule
\multicolumn{2}{>{\bfseries}c}{\textit{RentalManager $\longrightarrow$  assignDiscount(rental)}}\\
\toprule

Parameter is null & A NullArgumentException is raised\\
\midrule
Invalid parameter : not finished rental & A IllegallArgumentException is raised\\
\midrule
Valid parameter : finished rental & Assigns to the rental the list of the possible discount that user may receive. Now rental has a list (empty or with one or more value) containing discount values.\\

\bottomrule
\end{tabularx}
\end{center}

\begin{center}
\begin{tabularx}{\columnwidth}{>{\bfseries}XX}
\toprule
\multicolumn{2}{>{\bfseries}c}{\textit{PaymentManager $\longrightarrow$  computeKeepAsidePrice(rental)}}\\
\toprule

Parameter is null & A NullArgumentException is raised\\
\midrule
Valid parameter: rental with empty List of keepAside & Returns 0 (zero)  \\
\midrule
Valid parameter: rental with no empty List of KeepAside & Returns the total cost of all keepAsides: this cost is computed as the price per minutes of keepAside times the duration in minutes \\
\bottomrule
\end{tabularx}
\end{center}

\begin{center}
\begin{tabularx}{\columnwidth}{>{\bfseries}XX}
\toprule
\multicolumn{2}{>{\bfseries}c}{\textit{PaymentManager $\longrightarrow$  computeRentalPrice(rental)}}\\
\toprule

Parameter is null & A NullArgumentException is raised \\
\midrule
Invalid parameter: a rental is not ended yet & An IllegalArgumentException is raised \\
\midrule
Valid parameter: ended rental & Returns only the cost of the rental: this cost is computed as the price per minutes of rental times the duration in minutes\\
\bottomrule
\end{tabularx}
\end{center}

\begin{center}
\begin{tabularx}{\columnwidth}{>{\bfseries}XX}
\toprule
\multicolumn{2}{>{\bfseries}c}{\textit{PaymentManager $\longrightarrow$  computeTotalPrice(rental)}}\\
\toprule

Parameter is null & A NullArgumentException is raised\\
\midrule
Invalid parameter: a rental is not ended yet & An IllegalArgumentException is raised\\
\midrule
Valid parameter: ended rental & Returns the total cost of the rental $\longrightarrow$ sum rental cost and keepAside cost and applies discounts: it add the maximum value of discounts (overcharge) and the minimum value of discounts (best discount)\\

\bottomrule
\end{tabularx}
\end{center}

\begin{center}
\begin{tabularx}{\columnwidth}{>{\bfseries}XX}
\toprule
\multicolumn{2}{>{\bfseries}c}{\textit{PaymentManager $\longrightarrow$  addBill(rental)}}\\
\toprule
Parameter is null & A NullArgumentException is raised\\
\midrule
Invalid parameter: a rental is not ended yet & An IllegalArgumentException is raised \\
\midrule
Valid parameter: ended rental & Creates a bill of the rental and writes it to the database; then notifies the amount of the bill to the external Payment Handler\\
\bottomrule
\end{tabularx}
\end{center}

\section{Reservation}

\begin{center}
\begin{tabularx}{\columnwidth}{>{\bfseries}lX}
\toprule
Test case identifier & I7T1\\
\midrule
Test items & ReservationManager $\longrightarrow$ CarsManager, UserManager\\
\midrule
Input specification & Call to the addReservation method\\
\midrule
Output specification & The user is set to busy, the reservation is stored in the database, the timeout is triggered, the chosen car is correctly reserved\\

\bottomrule
\end{tabularx}
\end{center}

Below the description of the involved method calls.

\begin{center}
\begin{tabularx}{\columnwidth}{>{\bfseries}XX}
\toprule
\multicolumn{2}{>{\bfseries}c}{\textit{ReservationManager $\longrightarrow$  startTimeout(reservation)}}\\
\toprule

No parameter & Starts the hour-long countdown \\
\bottomrule
\end{tabularx}
\end{center}

\begin{center}
\begin{tabularx}{\columnwidth}{>{\bfseries}XX}
\toprule
\multicolumn{2}{>{\bfseries}c}{\textit{ReservationManager $\longrightarrow$  addReservation(user, car)}}\\
\toprule

One or more parameters are null & A NullArgumentException is raised\\
\midrule
Car is not available (checkAvailability returns false) & An InvalidRequestException is raised, car is unavailable \\
\midrule
User is already busy & An InvalidRequestException is raised, user is renting or reserving a car in this moment \\
\midrule
Set of valid parameter & Creates an instance of reservation, calls setState of car to change it (from available state to reserve state), then the reservation is written to the database \\
\bottomrule
\end{tabularx}
\end{center}

\section{End reservation}

\begin{center}
\begin{tabularx}{\columnwidth}{>{\bfseries}lX}
\toprule
Test case identifier & I8T1\\
\midrule
Test items & ReservationManager $\longrightarrow$ RentalManager; RentalManager $\longrightarrow$ OnBoardSystemManager, CarsManager;\\
\midrule
Input specification & A call to getReservation\\
\midrule
Output specification & The car is now in a rental state, with the associate unlock code, ready to be unlocked\\
\bottomrule
\end{tabularx}
\end{center}

Below the description of the involved method calls.

\begin{center}
\begin{tabularx}{\columnwidth}{>{\bfseries}XX}
\toprule
\multicolumn{2}{>{\bfseries}c}{\textit{ReservationManager $\longrightarrow$  getReservation(user)}}\\
\toprule

Parameter is null & A NullArgumentException is raised\\
\midrule
Invalid parameter: the user is not involved in any reservation& An IllegalArgumentException is raised\\
\midrule
Valid parameter & Gets the reservation related to the user \\
\bottomrule
\end{tabularx}
\end{center}

\begin{center}
\begin{tabularx}{\columnwidth}{>{\bfseries}XX}
\toprule
\multicolumn{2}{>{\bfseries}c}{\textit{ReservationManager $\longrightarrow$ endReservation(reservation)}}\\
\toprule

Parameter is null & A NullArgumentException is raised\\
\midrule
Invalid parameter: reservation is already finished & An IllegalArgumentException is raised\\
\midrule
Valid parameter & Set the time that reservation is finished and update database value.\\
\bottomrule
\end{tabularx}
\end{center}


\section{Registration}
This is a supplementary test description which refers to the unit tests to be carry out on the User Manager.
\begin{center}
\begin{tabularx}{\columnwidth}{>{\bfseries}XX}
\toprule
\multicolumn{2}{>{\bfseries}c}{\textit{UserManager $\longrightarrow$ verifyAlreadyRegistered(name, surname, mail, cf)}}\\
\toprule
One or more parameters are null &A NullArgumentException is raised \\
\midrule
Mail already used & Return false, user already registered \\
\midrule
CF already used & Return false, user already registered \\
\midrule
An incorrect mail address or name or surname & An InvalidArgumentException is raised \\
\midrule
An incorrect CF (no related person) & An InvalidArgumentException is raised \\
\midrule
Set of valid parameters & Return true, registration can continue and verify driverLicense and code Account \\
\bottomrule
\end{tabularx}
\end{center}

\begin{center}
\begin{tabularx}{\columnwidth}{>{\bfseries}XX}
\toprule
\multicolumn{2}{>{\bfseries}c}{\textit{UserManager $\longrightarrow$  verify(driverLicense,codeAccount)}}\\
\toprule
One or more parameters are null & A NullArgumentException is raised\\
\midrule
DriverLicense already used & Return false, user already registered\\
\midrule
Incorrect driverLicense or code account & An InvalidArgumentException is raised\\
\midrule
Set of valid parameters & Return true, registration is finished and user can be added to the system\\
\bottomrule
\end{tabularx}
\end{center}

\begin{center}
\begin{tabularx}{\columnwidth}{>{\bfseries}XX}
\toprule
\multicolumn{2}{>{\bfseries}c}{\textit{UserManager $\longrightarrow$  generatePassword()}}\\
\toprule
No input parameter & Create and return a password for new user \\
\bottomrule
\end{tabularx}
\end{center}

\begin{center}
\begin{tabularx}{\columnwidth}{>{\bfseries}XX}
\toprule
\multicolumn{2}{>{\bfseries}c}{\textit{UserManager $\longrightarrow$  addUser(name, surname, mail, cf, driverLicense, codeAccount)}}\\
\toprule

One or more parameters are null & A NullArgumentException is raised \\
\midrule
One or more parameters are incorrect  ( verifyAlreadyRegistered  or verify returns false ) & An InvalidRegistrationException is raised, user is already registered \\
\midrule
One or more parameters are incorrect  ( verify or verifyAlreadyRegistered raise an exception ) & Raised exception is propagated, user inserts incorrect parameter \\
\midrule
Set of valid parameters ( verify and verifyAlreadyRegistered returns true ) & Create a new user with parameters and password obtained from generatePassword and then user is added to the user's list and is written in the database \\
\bottomrule
\end{tabularx}
\end{center}


\begin{comment}
\begin{center}
\begin{tabularx}{\columnwidth}{>{\bfseries}XX}
\toprule
\multicolumn{2}{>{\bfseries}c}{\textit{UserManager $\longrightarrow$ verifyAlreadyRegistered(name, surname, mail, cf)}}\\
\toprule

\bottomrule
\end{tabularx}
\end{center}
\end{comment}


\chapter{Tools and test equipment required}
\subsection{Tools}
Tests are a significant key to produce stable and reliable systems, but to be effective they have to be carried out thoroughly. The best way to reduce the efforts, focusing on the goals, is to rely on tools optimized for that purpose.

Our advice is to adopt the \emph{JUnit} framework flanked by the \emph{Arquillian} integration testing framework, because of the Java environment that characterizes the system components.

\emph{Arquillian} is an innovative and highly extensible testing platform for the JVM that enables developers to easily create automated integration, functional and acceptance tests for Java middleware. This tool will help to verify that components interact each other in the designed way, producing the expected behaviors.
In particular, 
Because of our experience and based also on the guidelines of the tool, to add the framework to the system is better to work with \emph{Apache Maven}.

\emph{JUnit} is a proven framework which allows to deal with test cases gracefully, directly in the developing environment, to speed up and simplify developers' and testers' life. Then it is clear that this tool will be used to support tests by Arquillian.

To perform the unit tests, we suggest to adopt the \emph{Mockito} framework to take advantage of its flexibility and its power in the simulation of objects and behaviors of classes and methods.

\chapter{Program stubs and test data required}
\section{Program stubs and drivers}
The integration strategy we chose does not require as much stubs or drivers as the bottom-up or top-down strategies.
A stub needed is the one which simulates the external Payment Handler behavior.
We also need drivers to test the \emph{DataManager}, which simulates the behavior of the components in the writing and reading operations.

\section{Test Data}
Some test data are required to tackle the planned tests, and of course this data have to especially bring to light the plight of the test cases.

When stated ``fair'' is intended that there are at least one data which satisfies the condition required and at least a bunch of data that are considered safe by the test designers (that is, supposed to be regular, standard, correct).
Here below a list to be complied:
\begin{itemize}

\item A fair amount of data related to the Map thread, including instances presenting:
	\begin{itemize}
	\item null object;
	\item null fields;
	\item invalid positions;
	\item invalid cars;
	\item available cars;
	\item unavailable cars;
	\item PowerGrid stations;
	\item SafeAreas;
	\end{itemize}

\item A fair amount of data related to the Issue thread, including instances presenting:
	\begin{itemize}
	\item null object;
	\item null fields;
	\item invalid man;
	\item invalid car states;
	\item invalid issue states;
	\item invalid phone numbers;
	\item invalid description;
	\end{itemize}

\item A fair amount of data related to the Rental thread, including instances presenting:
\begin{itemize}
	\item null object;
	\item null fields;
	\item absurd plate number;
	\item unavailable cars;
	\item user involved in a rental yet;
	\end{itemize}

\item A fair amount of data related to the Unlock \& start rental, including instances presenting:
\begin{itemize}
	\item null object;
	\item null fields;
	\item illegal code;
	\item terminated rental;
	\item available car;
	\end{itemize}

\item A fair amount of data related to the Keep aside, including instances presenting:
\begin{itemize}
	\item null object;
	\item null fields;
	\item terminated rental;
	\item available car;
	\end{itemize}

\item A fair amount of data related to the End rental, including instances presenting:
\begin{itemize}
	\item null object;
	\item null fields;
	\item terminated rental;
	\item a non-terminated rental;
	\end{itemize}

\item A fair amount of data related to the Reservation, including instances presenting:
\begin{itemize}
	\item null object;
	\item null fields;
	\item unavailable car;
	\item busy user;
	\item invalid reservation;
	\end{itemize}

\item A fair amount of data related to the End reservation, including instances presenting:
\begin{itemize}
	\item null object;
	\item null fields;
	\item available car;
	\item invalid reservation;
	\end{itemize}
\end{itemize}

\chapter{Effort spent}

\begin{center}
    \begin{tabular}{cc}
        \toprule
	   \textbf{ Marco [h]  }& \textbf{ Daniele [h] }\\
	   \midrule
        26&22\\
	\bottomrule
    \end{tabular}
\end{center}
\chapter{References}
To draw up this document, we refer to the sample Intergation Test Plan Document provided in the lectures.
\end{document}
















